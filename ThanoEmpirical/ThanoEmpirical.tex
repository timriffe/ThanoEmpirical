%%This is a very basic article template.
%%There is just one section and two subsections.
\documentclass{article}
\usepackage{appendix}
\usepackage{amsmath}
\usepackage{caption}
\usepackage{placeins}
\usepackage{graphicx}
\usepackage{subcaption}
\usepackage{longtable}
\usepackage{tikz}
%\usepackage[active,tightpage]{preview}
\usepackage{natbib}
\bibpunct{(}{)}{,}{a}{}{;} 
\usepackage{url}
\usepackage{nth}
\usepackage{authblk}
% for the d in integrals
\newcommand{\dd}{\; \mathrm{d}}
\newcommand{\tc}{\quad\quad\text{,}}
\newcommand{\tp}{\quad\quad\text{.}}

% working on this need to concatenate file name based on sex and variable name
\newcommand\Cell[1]{{\raisebox{-0.05in}{\includegraphics[height=.2in,width=.2in]{Figures/ColorCodes/\expandafter#1}}}}  

\defcitealias{HMD}{HMD}

\begin{document}

\title{Time-to-death patterns in markers of age and dependency}

\author[1]{Tim Riffe\thanks{triffe@demog.berkeley.edu}}
\author[1]{Pil H. Chung}
\author[2,3]{Jeroen Spijker}
\author[4]{John MacInnes}
\affil[1]{Department of Demography, University of California, Berkeley}
\affil[2]{Department of Geography, Universitat Aut{\`o}noma de Barcelona}
\affil[3]{Vienna Institute of Demography}
\affil[4]{School of Social and Political Science, University of Edinburgh}

\maketitle

\begin{abstract}
We aim to determine the extent to which variables commonly
used to describe health, well-being, and disability in old-age vary primarily
as a function of years lived (age), years left, or as a function of both. We analyze data from the US Health and Retirement Study to estimate
chronological age and time-to-death patterns in 78 such variables. We describe
results from the birth cohort born 1915-1919 in the final 12 years of life of
individuals that died between 1992 and 2011 (ages 72 to 95). Our results show
that most markers used to study well-being in old-age vary along both the age
and time-to-death dimensions, but some markers are exclusively a function of
either time to death or chronological age, and others display different patterns
between the sexes.
\end{abstract}

\section*{Background}

For an
individual, age across the lifecourse is comprised two components: time since
birth and time to death, the \textit{chronological} and \textit{thanatalogical}
dimensions of age, respectively. In the aggregate, thanatological age is determined
by the mortality rate schedule to which a birth cohort is subject until its
extinction. Individuals do not know their thanatological age with certainty. To
guess this quantity one projects an expectation of lifespan based on scenarios
or extrapolations of how mortality rates might change over time. Data classified by chronological age, like census population counts, can be
reclassified into thanatological age in this way.\footnote{A paper on this topic
is currently under review \citep{riffe2014paaposter}. \citet{brouard1986structure, brouard1989mouvements} had already done the same thing some 30 years earlier, and we understand that
S. Scherbov also unwittingly produced the same result over a decade ago.}
Prospectively, decreasing mortality has the effect of moving population into higher thanatological ages, thereby increasing
remaining life expectancy \citep{sanderson2005average}. In this case,
the notion and measure of future remaining lifespan is elastic, subject to uncertainty.
In retrospect (after the death of a cohort), the thanatological age structure of
a population is a fixed characteristic. Furthermore, since a closed birth cohort
is akin to a stationary population,\footnote{The age structure of a birth cohort over time is proportional to the $l(x)$ column of the lifetable that describes its
mortality, which is proportional to the stable age structure determined by
the Lotka-Euler renewal model when the intrinsic growth rate is equal to zero.}
the chronological and thanatological age profiles are identical
\citep{brouard1989mouvements,vaupel2009life,rao2014generalization}. Yet, even in
the case of stationary populations, the age profiles of other demographic characteristics
in the population are decidedly different when viewed chronologically versus
thanatologically. Distinct patterns emerge in the aggregate due to an interaction between lifespan variation and the age profile(s) of
demographic characteristics.\footnote{When we state that a characteristic is a
function of either age perspective we do not imply that age causes the given
characteristic to vary, but rather that a characteristic varies in some smooth, regular, or parsimonious
way over age.}

Some life
transitions, states, and changes in state intensities are almost exclusively a
function of time to death. There are also markers where chronological age
captures almost all pertinent variation. In cases where a characteristic strongly varies as a
function of time to death (e.g., increases exponentially with the approach to death),
the common practice of aggregation over chronological age may misrepresent time
trends and misguide analyses about change over time and expectations for the
future. Measurment of the
end-of-life trajectories of characteristics is useful in such cases as a way of separating
trends in characteristics with time-to-death patterns from trends in mortality.
Characteristic measurements are taken while the respondent is alive, but
thanatological age at each interview is unknown until the date of death is
known, and is therefore retrospectively assigned. This analytical step lends clarity to the understanding of how characteristics vary over the lifespan and over time.

Incorporating a time-to-death perspective in demographic studies is especially
important when assessing the impact of ``population ageing''.
To the extent that the health, welfare, and social care demands of a
population are functions of thanatological rather than chronological age
structure, forecasts of the social and economic ``costs'' of ageing that are
based only on chronological age profiles are prone to misinterpretation.
In our concluding discussion we return to this point.

Research exploring time-to-death patterns has been done in other
domains, and topics examined can be roughly categorized into two types: 1) things that are a
function of apparent or perceived time to death
\citep{hamermesh1985expectations,hurd1995evaluation,carstensen2006influence,gan2004subjective,biro2010subjective,salm2010subjective,van2010living,cocco2012longevity,payne2013life,balia2013survival},
and 2) things that are a function of actual time to death
\citep{miller2001increasing,seshamani2004longitudinal,werblow2007population}.
The first kind are mostly studies on cognitive transitions and economic or
health behaviors, while the second kind are mostly studies on health
expenditure.
Another branch of research relates perceived and actual remaining lifetime
\citep{perozek2008using,delavande2011differential,post2012longevity,kutlu2013individuals}.
In this paper we will expand the second group, focusing on a broad range of
questions from ten waves of the US Health and Retirement Study \citep{HRS}.

%Data classified by chronological age, like census population counts, can be
%reclassified into thanatological age using some simple lifetable projection
%assumptions.\footnote{A paper on this topic is currently under review
%\citep{riffe2014paaposter}. \citet{brouard1986structure, brouard1989mouvements}
% had already done the same thing some 30 years earlier, and we understand that S. Scherbov also unwittingly produced the same result over a decade ago.}
%This transformation yields the thanatological age structure of the population
%under a particular set of mortality assumptions.
%Such lifetable-based transformations are idealized, and in many informal
%conversations the question has been raised as to which life transitions
%may actually be better understood as a function of thanatological
%age rather than chronological age.

%The present study is exploratory. We have the impression
%that such patterns are mostly novel to demographers and as-yet unincorporated
% in population-level indicators of ageing or disability. 

We aim to categorize domains in which it makes
sense to think from a thanatological perspective or the chronological
perspective, especially around various dimensions of morbidity associated with
the ageing process. Our analytical approach is retrospective rather than
prospective, meaning that no lifetable assumptions are made in the measurement
of thanatological age, and no censoring adjustments are necessary. Although more
data are available for adjacent cohorts, we report results only for the cohort
born from 1915 to 1919, which contains the longest set of observations in the
dataset used.

\section*{Data \& Method}

All findings reported in this paper are based on data from the US Health and
Retirement Study (HRS). We use the Rand edition of the data, which is
conveniently merged across all ten waves. This data is free to download, and all
details of data processing and methods are made freely available in an open code
repository.\footnote{This
repository includes R code used to process data, as well as generate results and
figures: \url{https://github.com/timriffe/ThanoEmpirical}} We
calculate thanatological age based on date of death information in the mortality followup module. We restrict the sample to only those individuals born between 1900 and 1930 that died between 1992 and 2011, which narrows the dataset down to 37051 observations of 9238 individuals. 8137 of these observations are from the 1919 individuals that died from the 1915-1919 cohort. Adjacent cohorts are kept for the sake of a
smoother loess fit to the data, which we describe in the following paragraphs.

\begin{figure}[!h]
\centering
\caption{Years lived and years left over the lifespan of a birth cohort}
\label{fig:LexisOrtho}
	\includegraphics{Figures/LexisOrtho.pdf}
\end{figure}

Underpinning this investigation are a series of heat maps indicating the
average intensity of a given marker along the chronological and thanatological
time dimensions within a series of quinquennial birth cohorts. This visual tool
is similar to but orthogonal to the familiar Lexis surface.
Figure~\ref{fig:LexisOrtho} orients the reader within the temporal coordinates we use. This diagram represents the various poissible lifespans within a given
birth cohort, with an arbitrary final age, $\omega$, of 110.
Members of the birth cohort are born on the left side of the diagram, at
chronological age zero and with an unknown $y$ coordinate (remaining lifetime)
at the time of birth.
Lifelines advance downward and to the right, where the downward direction indicates the approach to death, and the
rightward direction represents both the progression of calendar years and
chronological age. The blue arrow (B) indicates a hypothetical lifeline that
will eventually expire at age 99, although this property is unknown until death. The
present study contains only complete lifelines, such as that depicted in the
color red (A) in Figure~\ref{fig:LexisOrtho}. In this diagram, diagonal lines
represent death cohorts, rather than the birth cohorts found in the standard
Lexis diagram.

We limit the current study to the 1915-1919 cohort due
to the characteristics of the data source. Using the HRS, enough
observations are available so that we can measure the patterns
within the area outlined in green (C) in Figure~\ref{fig:LexisOrtho}. The
left bound of this area is chronological age 72, which the diagonal right
bound belongs to the completed lifespan of 95. While there are some observations
at thanatological ages greater than 12, there are too few to produce reliable estimates, and this choice of upper bound is somewhat arbitrary. Since the HRS spans 20 calendar years (1992-2011), the theoretical
upper bound of observation for thanatological age is 20. Future waves will
expand the area applicable to all but the oldest birth cohorts that are already
extinct in the data.

\paragraph*{Age}
Thanatological age is calculated for each individual as the lag between
interview and death dates expressed as decimal years. Chronological age is
calculated as the lag between birth and interview date in decimal years. Each
individual is therefore assigned a chronological and thanatological at each
interview, along with our accompanying variables of interest.
Birth dates and death dates in the original data are typically rounded up to the end
of the month, but this is more than enough precision for our purposes. Since we
are interested in viewing characteristics over both chronological age and
thanatological age simultaneously, we require a spread of data over
the whole study area. 
The current HRS dataset runs from 1992 to 2011, which
means that each birth cohort is observed over a different range of ages. For
example, the 1925-1929 cohort enters observation in 1992 at age 62 (at the
youngest) and acheives a maximum completed age of 85 by the end of 2011. On the
other end, the 1905-1909 enters the HRS in 1992 at age 78 at the youngest and
has a maximum completed lifespan of 105 by the last wave in 2011, albeit with
few observations at the upper extreme. Results from these and other birth cohorts are also
valid, but portions of these surfaces are based on fewer data points (lifespans $>$ 100) or ages in which labor market
exits appear to drive patterns at least as much as senescence (ages $<$ 67,
approximately). We selected the 1915-1919 cohort because
its observation window is centered on the chronological ages in which most
deaths occur,\footnote{The modal ages at death for the 1915-1919 cohort are
80-81 for males and around 87 for females (calculations based
on partially observed cohort $M(x)$ \citep{HMD}).} and because the HRS
provides a good density and spread of data points over this window. The lower
and upper age bounds may vary if questions were not available in the first,
second or final waves.


%Figure~\ref{fig:counts} shows the case count in two-year age groups over the
%range of ages that are in the data and included for this study (yellow bin).
%Darker blues indicate higher case counts, and the contours approximate counts
% at this level of aggregation. In practice, these counts are a maximum, since
%particular variables studided may be missing.

%\begin{figure}[!h]
%\centering
%\caption{Case counts in two-year age bins. Chronological age (Years lived) on
%the x axis and thanatological age (Years lives) on the y axis. Darker blues
%indicate higher case counts.}
%\label{fig:counts}
%\begin{subfigure}{\linewidth}
%	\caption{Males}
%	\vspace{-1em}
%	\label{fig:MalesCases}
%	\includegraphics[scale=.7]{Figures/CaseCountMales.pdf}
%\end{subfigure}
%\\
%\begin{subfigure}{\linewidth}
%    \caption{Females}
%   \vspace{-1em}
%	\label{fig:FemalesCases}
%    \includegraphics[scale=.7]{Figures/CaseCountFemales.pdf}
%\end{subfigure}
%\end{figure}

%Note that Figure~\ref{fig:counts} resembles a Lexis surface in some ways, but
% it is not organized by years or birth cohorts. Instead, years and birth cohorts in
%the data are overlapped and treated as a single period and a single birth
%cohort. We limit the study area to chronological ages 70 and higher, and
%thanatological ages 15 and lower, such that the sum of the two ages does not
%exceed 100. These bounds allow for reasonably stable estimates of surfaces for
%the studied characteristics. We cut off below chronological age 70 in order to
%remove some patterns that appear to be due to retirement rather than
%senescence processes, and also to avoid some potential compositional bias, as
%these are the typical ages of recruitment for the HRS.\footnote{Conceivably,
% the study area could be expanded after the addition of future mortality follow-ups.} 

\paragraph*{Characteristics}
We aim for a broad overview of the age variation across different dimensions of
old-age disability and well-being. For this reason we select a wide variety of
questions from the HRS data. These include questions
grouped roughly into the following categories: Activities of daily living (ADL),
instrumental activities of daily living (IADL), measures of healthcare utilization, functional and chronic conditions, psychological
measures, and health behaviors. We do not include charachteristics that were
not measured continously over all survey waves, or measurements that were
introduced to the survey after wave three. In all, we summarize results from 78
individual and composite items. We discarded variables that were not asked
continuously from at least wave 3 through 9. Variables not available in the
first or second wave have lower age bounds at higher ages than 72, whereas items
not asked in wave ten have upper lifespan bounds that are lower than 95.

Each survey question must be in a format suitable for numeric operations.
This requires some compromises in data quality, since some coded responses are less directly
quantifiable, and our translation of qualitative or ordinal responses to numeric
values was at times ad-hoc. In most cases, responses are easy to imagine as
``good'' and ``bad'', in which case we assigned higher values to ``bad'' and
lower values to ``good'' outcomes. For example, respondents were asked if they
felt depressed. We assigned 0 to answers of ``no'' and 1 to answers of ``yes''.
Rather than divide all questions into binary responses, we assigned intermediate
values on a case by case basis. For example, self-reported health
had possible responses of ``excellent'', ``very good'', ``good'',
``fair'', and ``poor'', which we assigned values in equal intervals
from 0 to 1, respectively. Some response
sets for particular questionnaire items changed between
waves.
In these cases, we attempted to assign numerical codes that were consistent
over the transition. These choices are far from clean, but they are good enough
to meet the goals of this study.\footnote{The pre-processing of variables is
full of details that would clutter this paper. Rather than a lengthy and detailed appendix describing the case by case treatment of variables, we make the full code base used in the generation of results available in a github repository.}

Variables with compact or bounded numeric responses were rescaled to range from
0 to 1. Variables with no clear bounds or very large upper
bounds, such as medical expenditure, body mass index, or number of hospital
visits were not rescaled. These rescalings are intended to simplify
the visual interpretation of surfaces, and they do not alter the
quantitative summary measures we use later.

\paragraph*{Weighting}
The population universe of the HRS and this study is the resident
population of the United States. Therefore person weights are needed in
order to estimate population-level means. 
One difficulty with the HRS is that the institutionalized population is treated
as a second target population. In all waves but 5 and 6, there are no person weights
assigned to institutionalized individuals. We try to impute missing
person-weights according to some simple assumptions. If the individual was
assigned a weight in a previous wave, we carry this weight over as a constant, unless there was also a non-zero weight in a future interview, in
which case we assign the weight according to the within-individual linear
pattern. Individuals and interviews that still have missing person-weights
after this procedure are discarded from our study. We perform no other
within-individual interpolation of observations.

\paragraph*{Loess smoothing}
Direct tabulations of the weighted data are legibile if all birth
cohorts are combined, but doing this distorts results due to cohort
composition bias. To overcome birth cohort heterogeneity within surfaces,
we use birth cohorts as a third time dimension. Tabulations within this three dimensional space are noisy, and so we
enhance surface legibility by using a non-parametric local smoother.
We specify a loess model of the given characteristic over chronological age,
thanatological age, and quinquennial birth cohorts, using all observations of since-deceased individuals from the 1900 through the 1934 birth cohorts. We fit the model using the \texttt{loess()} function in base \texttt{R} \citep{cleveland1992local,Rcore2013}\footnote{Using the fitted model, surfaces are produced using the related loess prediction function, \texttt{predict.loess()}. The smoothing parameter, \texttt{spar}, is set to 0.7 for the results we present in the paper.
All results were also produced using smoothing parameters of .5, and .9, and
we concluded that the specific choice of smoothness does not drive results.
The three predictor dimensions are not normalized, in order to preserve year
units. } to the weighted individual-level data for each sex separately, and then
predict a surface for the 1915-1919 birth cohort within the study area outlined
in green (C) in Figure~\ref{fig:LexisOrtho}. Weighting is therefore explicit by
person-weights, and implicit by point density within the three temporal
dimensions.

%\subsection{Finding the fall-line}


%In order to characterize this surface, we find the direction
%of steepest ascent for a regular grid of points. These directions are shown
% with arrows in Figure~\ref{fig:srh}. If the direction is exactly $90^\circ$ or $270^\circ$,
%then we know that the process is essentially a thanatological one, at least in
%the age-range studied. If the direction is $0^\circ$ or $180^\circ$, then the
%characteristic can be said to be chronological. In our experience, most
%variables that we might be interested in are functions of both kinds of age,
% and so to determine the primary source of variation, we take the slope-weighted mean
%of each arrow direction, translate this to the upper-right unit quadrant, and
%then translate to a percent scale, where $90^\circ$ is 100\% thanatological and
%$0^\circ$ is 100\% chronological. For example, we judge SRH for males
%(Figure~\ref{fig:srh}) to be \%85.4 thanatological. However, around age 70-75,
%it is nearly \%100 thanatological.

%Some variables display complex surfaces, and may operate in different ways
%depending on particular age coordinates. In these cases, the result of the
% above procedure is too simple, and for this reason we provide the surface plots in supplementary online
%material.\footnote{Plots are for each sex separately, and have been compiled
%into two multipage pdf documents, each containing all variables for one sex.
%Links have been shortened.
%Males:
%\url{http://goo.gl/pzEHHk}, Females:
%\url{http://goo.gl/xyBMxO}.}



\section*{Results}

As examples, we present surfacs representing psychological problems for males
(Figure~\ref{fig:psych}) and back pain for males (Figure~\ref{fig:back}).
These two surfaces were selected from the full set of 78 characteristics because they cleanly fit the
description of thanatological and chronological characteristics, respectively. 

The model fit for each variable is used to produce a contour surface, which can
be interpreted visually. In most situations it is obvious to the eye whether a
variable operates over thanatological age or over chronological age, but there
are many instances where both are at play, or where the relationship is
complex. An example surface for self-reported health (SRH) is shown in
Figure~\ref{fig:srh}.
%\begin{figure}[!h]
%\centering
%\caption{Case counts in two-year age bins. Chronological age (Years lived) on
%the x axis and thanatological age (Years lives) on the y axis. Darker blues
%indicate higher case counts.}
%\label{fig:counts}
%\begin{subfigure}{\linewidth}
%	\caption{Males}
%	\vspace{-1em}
%	\label{fig:MalesCases}
%	\includegraphics[scale=.7]{Figures/CaseCountMales.pdf}
%\end{subfigure}
%\\
%\begin{subfigure}{\linewidth}
%    \caption{Females}
%   \vspace{-1em}
%	\label{fig:FemalesCases}
%    \includegraphics[scale=.7]{Figures/CaseCountFemales.pdf}
%\end{subfigure}
%\end{figure}
\begin{figure}[!h]
    % This figure was produced in
    % \makebox[\textwidth][c]{\includegraphics[scale=.6]{Figures/SurfExampleMalesSRH.pdf}}
    \centering
    \caption{Examples of characteristics that vary along the thanatological and
    chronological age axes.}
    \begin{subfigure}{\linewidth}
    \caption{Psychological problems (ever) by
    years lived (x axis) and years left (y axis). Males, 1915-1919 birth cohort.
    }
    \label{fig:psych}
	\vspace{-2em}
	\makebox[\textwidth][c]{\includegraphics[scale=.6]{Figures/Surf_Male_psych.pdf}}
	\end{subfigure}
	
	\begin{subfigure}{\linewidth}
    \caption{Back Problems by
    years lived (x axis) and years left (y axis). Males, 1915-1919 birth
    cohort.}
    \label{fig:back}
	\vspace{-2em}
	\makebox[\textwidth][c]{\includegraphics[scale=.6]{Figures/Surf_Male_back.pdf}}
	\end{subfigure}
\end{figure}
\FloatBarrier

In the case of Figure~\ref{fig:srh} note that there is regular variation in the
direction of both axes, but that the variation over thanatological age (y axis)
is sharper, i.e., it implies a steeper climb than does the apparent decrease
over chronological age. 

This section contains the distilled results for each sex-variable surface. In
each case we only report the average direction of steepest ascent, which tells
us the degree to which variation is primarily over thanatological age or over
chronological age. We do not display the strength of the
relationship.
A future revision will include detailed discussion of the results. For now, we
report that most variables studied vary much more over thanatological age
than over chronological age, at least for the age-ranges studied. The
consistency of this finding is much greater than we expected, but it is not
universal. 

Results for males and females are color-coded with a thermometer scale in order
to facilitate skimming the tables: dark red indicates
predominantly thanatological variation and dark blue indicates
predominantly chronological variation.

The most chronological variation in the data we examined are current and ever
smoking, body mass index (BMI) and having had outpatient surgery in the past 24
months. Most other variables are very strongly thanatological. We do not at the
time of this writing have an explanation for why smoking is not more
thanatolgical in nature, but one possibility is uncontrolled compositional
distortion in the underlying data. This could be so if younger respondents are,
on average, from later waves and older respondents are, on average, from earlier
waves, in which case strong birth cohort effects could carry through into this
otherwise \textit{timeless} age-surface. We will test for this possibility in a future
revision, and in this case we will make the same control for all variables. The
strength and consistency of the thanatological patterns obtained at this time is
nonetheless striking and appears worthy of more careful
investigation.

%\FloatBarrier
%\pagebreak
%\subsection{ADL}
%
%% latex table generated in R 3.0.1 by xtable 1.7-3 package
% Wed Oct 15 16:49:54 2014
\begin{table}[ht]
\centering
\begin{tabular}{p{6cm}rr}
  \hline
Question & Male \% Thano & Female \% Thano \\ 
  \hline
Diff. getting in/out bed  & \% 93.6 \Cell{adlbedMales.pdf} & \% 91.5 \Cell{adlbedFemales.pdf} \\ 
  Diff. eating  & \% 91.2 \Cell{adleatMales.pdf} & \% 92.2 \Cell{adleatFemales.pdf} \\ 
  Diff. using toilet & \% 88.0 \Cell{adltoiletMales.pdf} & \% 90.3 \Cell{adltoiletFemales.pdf} \\ 
  Diff. dressing  & \% 88.7 \Cell{adldressMales.pdf} & \% 88.7 \Cell{adldressFemales.pdf} \\ 
  ADL 5-point & \% 87.6 \Cell{adl5Males.pdf} & \% 88.9 \Cell{adl5Females.pdf} \\ 
  ADL 3-point & \% 86.9 \Cell{adl3Males.pdf} & \% 89.0 \Cell{adl3Females.pdf} \\ 
  Diff. walking across room  & \% 86.1 \Cell{adlwalkMales.pdf} & \% 85.8 \Cell{adlwalkFemales.pdf} \\ 
  Diff. bathing/showering  & \% 82.1 \Cell{adlbathMales.pdf} & \% 86.5 \Cell{adlbathFemales.pdf} \\ 
   \hline
\end{tabular}
\end{table}

%
%\FloatBarrier
%\subsection{IADL}
%% latex table generated in R 3.1.2 by xtable 1.7-4 package
% Wed Nov 19 21:45:14 2014
\begin{table}[ht]
\centering
\begin{tabular}{p{6cm}rr}
  \hline
Question & Male \% Thano & Female \% Thano \\ 
  \hline
Health limits work & \% 92.9 \Cell{limworkMales.pdf} & \% 91.5 \Cell{limworkFemales.pdf} \\ 
  Diff. taking meds & \% 91.0 \Cell{iadlmedsMales.pdf} & \% 85.3 \Cell{iadlmedsFemales.pdf} \\ 
  Diff. preparing hot meals & \% 86.3 \Cell{iadlmealsMales.pdf} & \% 86.8 \Cell{iadlmealsFemales.pdf} \\ 
  IADL 5-point & \% 85.6 \Cell{iadl5Males.pdf} & \% 85.8 \Cell{iadl5Females.pdf} \\ 
  IADL 3-point & \% 86.5 \Cell{iadl3Males.pdf} & \% 84.6 \Cell{iadl3Females.pdf} \\ 
  Diff. managing money & \% 87.6 \Cell{iadlmoneyMales.pdf} & \% 82.6 \Cell{iadlmoneyFemales.pdf} \\ 
  Diff. using telephone & \% 78.3 \Cell{iadltelMales.pdf} & \% 83.2 \Cell{iadltelFemales.pdf} \\ 
  Diff. shopping for groceries & \% 79.6 \Cell{iadlshopMales.pdf} & \% 80.6 \Cell{iadlshopFemales.pdf} \\ 
  Diff. using map & \% 84.0 \Cell{iadlmapMales.pdf} & \% 76.1 \Cell{iadlmapFemales.pdf} \\ 
   \hline
\end{tabular}
\end{table}

%\FloatBarrier
%
%\pagebreak
%\subsection{Chronic conditions}
%% latex table generated in R 3.0.1 by xtable 1.7-3 package
% Wed Oct 15 16:49:55 2014
\begin{table}[ht]
\centering
\begin{tabular}{p{6cm}rr}
  \hline
Question & Male \% Thano & Female \% Thano \\ 
  \hline
Psych problems , ever & \% 90.3 \Cell{psychMales.pdf} & \% 87.4 \Cell{psychFemales.pdf} \\ 
  Cancer, ever & \% 88.9 \Cell{cancerMales.pdf} & \% 86.6 \Cell{cancerFemales.pdf} \\ 
  Arthritis, ever  & \% 82.2 \Cell{arthMales.pdf} & \% 92.9 \Cell{arthFemales.pdf} \\ 
  Nr chronic conditions & \% 84.6 \Cell{ccMales.pdf} & \% 90.2 \Cell{ccFemales.pdf} \\ 
  Stroke, ever  & \% 81.8 \Cell{strokeMales.pdf} & \% 89.1 \Cell{strokeFemales.pdf} \\ 
  Heart problems, ever  & \% 79.4 \Cell{heartMales.pdf} & \% 90.4 \Cell{heartFemales.pdf} \\ 
  Lung disease & \% 80.1 \Cell{lungMales.pdf} & \% 76.1 \Cell{lungFemales.pdf} \\ 
  High blood pressure, ever & \% 74.3 \Cell{bpMales.pdf} & \% 78.9 \Cell{bpFemales.pdf} \\ 
  Diabetes, ever  & \% 66.3 \Cell{diabMales.pdf} & \% 66.6 \Cell{diabFemales.pdf} \\ 
   \hline
\end{tabular}
\end{table}

%
%\FloatBarrier
%
%\subsection{Functional limitations}
%% latex table generated in R 3.1.2 by xtable 1.7-4 package
% Wed Nov 19 21:45:14 2014
\begin{table}[ht]
\centering
\begin{tabular}{p{6cm}rr}
  \hline
Question & Male \% Thano & Female \% Thano \\ 
  \hline
Large muscle difficulty index & \% 92.0 \Cell{lgmusMales.pdf} & \% 94.2 \Cell{lgmusFemales.pdf} \\ 
  Mobility difficulty index & \% 90.9 \Cell{mobMales.pdf} & \% 90.6 \Cell{mobFemales.pdf} \\ 
  Fine motor difficulty index & \% 88.5 \Cell{finemotMales.pdf} & \% 91.0 \Cell{finemotFemales.pdf} \\ 
  Gross motor difficulty index & \% 89.0 \Cell{grossmotMales.pdf} & \% 88.8 \Cell{grossmotFemales.pdf} \\ 
  Back problems & \% 83.2 \Cell{backMales.pdf} & \% 83.8 \Cell{backFemales.pdf} \\ 
  BMI & \% 56.7 \Cell{bmiMales.pdf} & \% 46.4 \Cell{bmiFemales.pdf} \\ 
   \hline
\end{tabular}
\end{table}

%
%\citet{woolf2003burden} describe four muskuloskeletal conditions as functions of
%chronological age.
%
%\FloatBarrier
%
%\subsection{Behaviors}
%% latex table generated in R 3.0.1 by xtable 1.7-3 package
% Wed Oct 15 16:49:55 2014
\begin{table}[ht]
\centering
\begin{tabular}{p{6cm}rr}
  \hline
Question & Male \% Thano & Female \% Thano \\ 
  \hline
Alcohol, ever  & \% 87.4 \Cell{alcevMales.pdf} & \% 86.1 \Cell{alcevFemales.pdf} \\ 
  Alcohol nr of days / week  & \% 75.8 \Cell{alcdaysMales.pdf} & \% 79.8 \Cell{alcdaysFemales.pdf} \\ 
  Alcohol nr drinks per drinking day  & \% 64.0 \Cell{alcdrinksMales.pdf} & \% 69.7 \Cell{alcdrinksFemales.pdf} \\ 
  Ever Smoker & \% 49.9 \Cell{smokeevMales.pdf} & \% 41.3 \Cell{smokeevFemales.pdf} \\ 
  Current Smoker & \% 35.1 \Cell{smokecurMales.pdf} & \% 21.9 \Cell{smokecurFemales.pdf} \\ 
   \hline
\end{tabular}
\end{table}

%\FloatBarrier
%
%\pagebreak
%\subsection{Psychological factors}
%% latex table generated in R 3.1.2 by xtable 1.7-4 package
% Wed Nov 19 21:45:14 2014
\begin{table}[ht]
\centering
\begin{tabular}{p{6cm}rr}
  \hline
Question & Male \% Thano & Female \% Thano \\ 
  \hline
Could not get going & \% 92.4 \Cell{cesdgoingMales.pdf} & \% 94.8 \Cell{cesdgoingFemales.pdf} \\ 
  Everything an effort & \% 93.1 \Cell{cesdeffMales.pdf} & \% 90.4 \Cell{cesdeffFemales.pdf} \\ 
  Depression score & \% 90.9 \Cell{cesdMales.pdf} & \% 92.4 \Cell{cesdFemales.pdf} \\ 
  Felt Depressed & \% 89.6 \Cell{cesddeprMales.pdf} & \% 90.2 \Cell{cesddeprFemales.pdf} \\ 
  Felt sad & \% 90.9 \Cell{cesdsadMales.pdf} & \% 87.3 \Cell{cesdsadFemales.pdf} \\ 
  Self-reported health & \% 85.6 \Cell{srhMales.pdf} & \% 91.1 \Cell{srhFemales.pdf} \\ 
  Sleep restless & \% 88.0 \Cell{cesdsleepMales.pdf} & \% 80.8 \Cell{cesdsleepFemales.pdf} \\ 
  Was happy & \% 79.8 \Cell{cesdhappyMales.pdf} & \% 84.1 \Cell{cesdhappyFemales.pdf} \\ 
  Enjoyed life & \% 72.4 \Cell{cesdenjoyMales.pdf} & \% 91.5 \Cell{cesdenjoyFemales.pdf} \\ 
  Felt lonely & \% 61.5 \Cell{cesdloneMales.pdf} & \% 86.7 \Cell{cesdloneFemales.pdf} \\ 
   \hline
\end{tabular}
\end{table}

%\FloatBarrier
%
%\subsection{Healthcare utilization}
%% latex table generated in R 3.1.2 by xtable 1.7-4 package
% Wed Nov 19 21:45:14 2014
\begin{table}[ht]
\centering
\begin{tabular}{p{6cm}rr}
  \hline
Question & Male \% Thano & Female \% Thano \\ 
  \hline
Imputed total medical expend. & \% 96.5 \Cell{medexpMales.pdf} & \% 95.3 \Cell{medexpFemales.pdf} \\ 
  log medical expenditure & \% 94.0 \Cell{medexplogMales.pdf} & \% 96.6 \Cell{medexplogFemales.pdf} \\ 
  Overnight hospital: 24 mo & \% 93.9 \Cell{hospMales.pdf} & \% 94.6 \Cell{hospFemales.pdf} \\ 
  Nr hospital stays: 24 mo & \% 94.6 \Cell{hospstaysMales.pdf} & \% 92.9 \Cell{hospstaysFemales.pdf} \\ 
  Number  nights in hospital: 24 mo & \% 94.9 \Cell{hospnightsMales.pdf} & \% 90.9 \Cell{hospnightsFemales.pdf} \\ 
  Home health care: 24 mo & \% 94.3 \Cell{hhcMales.pdf} & \% 91.1 \Cell{hhcFemales.pdf} \\ 
  Number Dr visits: 24 mo & \% 93.9 \Cell{docvisitsMales.pdf} & \% 89.5 \Cell{docvisitsFemales.pdf} \\ 
  Nr nursing home stays: 24 mo & \% 89.6 \Cell{nhstaysMales.pdf} & \% 90.5 \Cell{nhstaysFemales.pdf} \\ 
  Overnight stay nursing home: 24 mo & \% 89.6 \Cell{nhMales.pdf} & \% 89.5 \Cell{nhFemales.pdf} \\ 
  Nursing home at interview  & \% 88.9 \Cell{nhnowMales.pdf} & \% 88.6 \Cell{nhnowFemales.pdf} \\ 
  Nr nights nursing home: 24 mo & \% 85.2 \Cell{nhnightsMales.pdf} & \% 85.2 \Cell{nhnightsFemales.pdf} \\ 
  Special health fac visit: 24 mo & \% 74.3 \Cell{shfMales.pdf} & \% 85.3 \Cell{shfFemales.pdf} \\ 
  Prescription drugs regularly: 24 mo & \% 78.6 \Cell{medsMales.pdf} & \% 77.3 \Cell{medsFemales.pdf} \\ 
  Dr visit: 24 mo & \% 68.6 \Cell{docMales.pdf} & \% 70.5 \Cell{docFemales.pdf} \\ 
  Dental visit: 24 mo & \% 67.7 \Cell{dentMales.pdf} & \% 68.0 \Cell{dentFemales.pdf} \\ 
  Outpatient surgery: 24 mo & \% 46.7 \Cell{surgMales.pdf} & \% 47.5 \Cell{surgFemales.pdf} \\ 
   \hline
\end{tabular}
\end{table}

%\FloatBarrier
%
%\pagebreak
%\subsection{Cognitive functions}
%% latex table generated in R 3.1.2 by xtable 1.7-4 package
% Wed Nov 19 21:45:14 2014
\begin{table}[ht]
\centering
\begin{tabular}{p{6cm}rr}
  \hline
Question & Male \% Thano & Female \% Thano \\ 
  \hline
Memory prob. at interview & \% 89.7 \Cell{mprobMales.pdf} & \% 86.7 \Cell{mprobFemales.pdf} \\ 
  Memory prob. Ever & \% 90.0 \Cell{mprobevMales.pdf} & \% 79.7 \Cell{mprobevFemales.pdf} \\ 
  Memory compared to past & \% 72.9 \Cell{pastmemMales.pdf} & \% 84.5 \Cell{pastmemFemales.pdf} \\ 
  Naming day of week & \% 85.8 \Cell{namedwkMales.pdf} & \% 69.9 \Cell{namedwkFemales.pdf} \\ 
  Naming day of month & \% 80.5 \Cell{namedmoMales.pdf} & \% 67.9 \Cell{namedmoFemales.pdf} \\ 
  Naming scissors & \% 77.2 \Cell{namesciMales.pdf} & \% 67.5 \Cell{namesciFemales.pdf} \\ 
  Self-rated memory & \% 60.1 \Cell{srmMales.pdf} & \% 84.0 \Cell{srmFemales.pdf} \\ 
  Naming month & \% 80.6 \Cell{namemoMales.pdf} & \% 63.3 \Cell{namemoFemales.pdf} \\ 
  Backwards counting & \% 79.7 \Cell{c20bMales.pdf} & \% 54.2 \Cell{c20bFemales.pdf} \\ 
  Mental status summary & \% 81.2 \Cell{tmMales.pdf} & \% 52.0 \Cell{tmFemales.pdf} \\ 
  Vocabulary score & \% 79.3 \Cell{vocabMales.pdf} & \% 52.6 \Cell{vocabFemales.pdf} \\ 
  Naming year & \% 76.9 \Cell{nameyrMales.pdf} & \% 54.9 \Cell{nameyrFemales.pdf} \\ 
  Serial 7s & \% 78.3 \Cell{ssMales.pdf} & \% 51.0 \Cell{ssFemales.pdf} \\ 
  Naming VP & \% 73.6 \Cell{namevpMales.pdf} & \% 51.9 \Cell{namevpFemales.pdf} \\ 
  Naming president & \% 80.9 \Cell{namepresMales.pdf} & \% 33.9 \Cell{namepresFemales.pdf} \\ 
  Naming cactus & \% 50.2 \Cell{namecacMales.pdf} & \% 52.2 \Cell{namecacFemales.pdf} \\ 
  Immediate word recall & \% 56.7 \Cell{iwrMales.pdf} & \% 41.1 \Cell{iwrFemales.pdf} \\ 
  Total word recall & \% 55.8 \Cell{twrMales.pdf} & \% 34.3 \Cell{twrFemales.pdf} \\ 
  Delayed word recall & \% 55.2 \Cell{dwrMales.pdf} & \% 28.6 \Cell{dwrFemales.pdf} \\ 
   \hline
\end{tabular}
\end{table}

%
%\FloatBarrier
\bibliographystyle{plainnat}
  \bibliography{references} 
%  
%\pagebreak
%
\begin{appendices}
\section{Variable coding}
This appendix provides details of how individual variables were numerically
coded for the purpose of calculating surfaces. In many cases the numeric coding
is an arbitrary mapping of a qualitative variable. Results may vary under
different mappings, but we assume that the fundamental findings of this study
will be robust to more thoughtful recodings. Most question items lend themselves
to being spread over a spectrum from good to bad, in which case we give bad the
highest value and good the lowest value. When response universes are bounded, we
scale responses to fit within the range [0,1]. Neither of these
two transformations affects the determination of the degree to which a
characteristic is thanatological.
\end{appendices}




  
\end{document}