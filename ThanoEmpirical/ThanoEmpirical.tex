%%This is a very basic article template.
%%There is just one section and two subsections.
\documentclass{article}

\usepackage{amsmath}
\usepackage{caption}
\usepackage{placeins}
\usepackage{graphicx}
\usepackage{subcaption}
\usepackage{tikz}
%\usepackage[active,tightpage]{preview}
\usepackage{natbib}
\bibpunct{(}{)}{,}{a}{}{;} 
\usepackage{url}
\usepackage{nth}
\usepackage{authblk}
% for the d in integrals
\newcommand{\dd}{\; \mathrm{d}}
\newcommand{\tc}{\quad\quad\text{,}}
\newcommand{\tp}{\quad\quad\text{.}}
\defcitealias{HMD}{HMD}

\begin{document}

\title{Time-to-death patterns in markers or age and dependency}

\author[1]{Tim Riffe}
\author[1]{Paul Chung}
\author[2]{Jeroen Spijker}
\author[2]{John MacInnes}
\affil[1]{Department of Demography, University of California, Berkeley}
\affil[2]{School of Social and Political Science, University of Edinburgh}

\maketitle

Some individual life transitions are probably best described as a function of
time since birth (chronological age), while others are probably best described as a
function of time until death (thanatological age). These two ways of
measuring age are different because lifespans vary between individuals.
Recently, we presented some results of transforming
data classified by chronological age, like census population counts, into
thanatological age using some simple lifetable
assumptions\footnote{\citep{riffe2014paaposter} A paper on this topic is
currently under review. \citet{brouard1986structure, brouard1989mouvements}
had already done the same thing some 30 years earlier.}.
This transformation yields the thanatological age structure of the population,
which is interesting enough to justify itself. However, such lifetable-based
transformations are idealized and hypothetical, and in many informal
conversations the question has been raised as to which life history transitions
actually are more precisely described as a function of thanatological age than
in the conventional way. Work has been done on this topic in other domains, and
this work can roughly be divided into two domains: 1) things that are a function
of apparent time to death \citep{carstensen2006influence,gan2004subjective,salm2010subjective,van2010living},
and 2) things that are a function of actual time to death
\citep{miller2001increasing,seshamani2004longitudinal}. The former are mostly
studies on cognitive transitions and economic behaviors, while the latter are
mostly studies on health expenditure (of which there are many). In this paper we
will try to expand the latter group


\bibliographystyle{plainnat}
  \bibliography{references} 
  
\end{document}