%%This is a very basic article template.
%%There is just one section and two subsections.
\documentclass{article}

\usepackage{amsmath}
\usepackage{caption}
\usepackage{placeins}
\usepackage{graphicx}
\usepackage{subcaption}
\usepackage{tikz}
%\usepackage[active,tightpage]{preview}
\usepackage{natbib}
\bibpunct{(}{)}{,}{a}{}{;} 
\usepackage{url}
\usepackage{nth}
\usepackage{authblk}
% for the d in integrals
\newcommand{\dd}{\; \mathrm{d}}
\newcommand{\tc}{\quad\quad\text{,}}
\newcommand{\tp}{\quad\quad\text{.}}
\defcitealias{HMD}{HMD}

\begin{document}

\title{Time-to-death patterns in markers of age and dependency}

\author[1]{Tim Riffe}
\author[1]{Pil H. Chung}
\author[2,3]{Jeroen Spijker}
\author[4]{John MacInnes}
\affil[1]{Department of Demography, University of California, Berkeley}
\affil[2]{Department of Geography, Universitat Aut{\`o}noma de Barcelona}
\affil[3]{Vienna Institute of Demography}
\affil[4]{School of Social and Political Science, University of Edinburgh}

\maketitle

Some individual life transitions are probably best described as a function of
time since birth (chronological age), while others are probably best described as a
function of time until death (thanatological age). These two ways of
measuring age are different in the aggregate because lifespans vary between
individuals.
Recently, we presented some results of transforming
data classified by chronological age, like census population counts, into
thanatological age using some simple lifetable
assumptions\footnote{\citep{riffe2014paaposter} A paper on this topic is
currently under review. \citet{brouard1986structure, brouard1989mouvements}
had already done the same thing some 30 years earlier, and we understand that
S. Scherbov also unwittingly produced the same result over a decade ago. It is
safe to say that most demographers are still unaware of the method, however.}.
This transformation yields the thanatological age structure of the population
under a particular set of mortality assumptions, which is interesting enough to
justify itself.
However, such lifetable-based transformations are idealized and hypothetical, and in many informal
conversations the question has been raised as to which life transitions
may actually be better modelled or understood as a function of thanatological
age than in the conventional way.

Work has been done on this topic in other domains, and
topics examined can be roughly categorized into two types: 1) things that are a
function of apparent or perceived time to death
\citep{carstensen2006influence,gan2004subjective,salm2010subjective,van2010living}, and 2) things that are a function of actual time to death
\citep{miller2001increasing,seshamani2004longitudinal}. The former are mostly
studies on cognitive transitions and economic behaviors, while the latter are
mostly studies on health expenditure (of which there are many). In this paper we
will expand the latter group, focusing on a broad range of questions from the US
Health and Retirement Survey \citep{HRS}, which at present provides a fairly
good sample to provide aggregate results for the final 15 years of life.

The present study is exploratory. We have the impression
that such patterns are mostly novel to demographers and as-yet unincorporated in
population-level indicators of ageing or disability. We aim to identify a
range of domains in which it makes sense to think of processes from a
thanatological perspective. Some preliminary results are shown in
Figure~\ref{fig:proposalfigs} for three summary indicators and one direct survey
question. ADL measures\footnote{We tested various indices
(\texttt{adl3},\texttt{adl5},\texttt{iadl3},\texttt{iadl5}), and all showed the
same pattern.} and self-reported health show clear patterns of
accelerated increase with the approach to death. Depression increases for both
males and females with the approach to death, but there appear to be sex
differences in functional forms. Back pain shows no clear pattern for females,
and a possible spurious decline for males. In the complete paper we will present a
more structured survey and a distilled set of comparisons that we hope will
enrich discussion at the conference. 

\begin{figure}
\centering
\caption{Time to death patterns in four substantive indicators of ageing and
dependency.}
\label{fig:proposalfigs}
\begin{subfigure}[b]{.47\linewidth}
\centering
	\caption{Females}
	\label{fig:females}
	\includegraphics[scale=.45]{Figures/VariablePlots/ProposalFemales.pdf}	
\end{subfigure}
~
\begin{subfigure}[b]{.47\linewidth}
\centering
    \caption{Males}
	\label{fig:ales}
    \includegraphics[scale=.45]{Figures/VariablePlots/ProposalMales.pdf}
\end{subfigure}
\caption{All waves combined. All trends have been centered around zero and
rescaled by $\sigma$. \texttt{adl5} is an index of difficulty with common
activities. \texttt{srh} is self-reported health (higher = worse).
\texttt{back} is back pain (higher = more). \texttt{cesd} is a depression score
(higher = worse).}
\end{figure}
%from here



\bibliographystyle{plainnat}
  \bibliography{references} 
  
\end{document}