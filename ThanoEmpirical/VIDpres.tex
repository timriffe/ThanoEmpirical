
%%%%%%%%%%%%%%%%%%%%%%%%%%%%%%%%%%%%%%%%%
% Beamer Presentation
% LaTeX Template
% Version 1.0 (10/11/12)
%
% This template has been downloaded from:
% http://www.LaTeXTemplates.com
%
% License:
% CC BY-NC-SA 3.0 (http://creativecommons.org/licenses/by-nc-sa/3.0/)
%
%%%%%%%%%%%%%%%%%%%%%%%%%%%%%%%%%%%%%%%%%

%----------------------------------------------------------------------------------------
%	PACKAGES AND THEMES
%----------------------------------------------------------------------------------------

\documentclass{beamer}
\usepackage[utf8]{inputenc}
\usetheme{Copenhagen}
\usecolortheme{orchid}

\usepackage{graphicx} % Allows including images
\usepackage{booktabs} % Allows the use of \toprule, \midrule and \bottomrule in tables
%----------------------------------------------------------------------------------------
%	TITLE PAGE
%----------------------------------------------------------------------------------------

\title[TTD markers]{Time-to-death patterns in markers of age and dependency}
 
\author[Riffe et. al.]
{
T. Riffe \inst{1} \and P. H. Chung \inst{1} \and J. Spijker \inst{2} \and J.
MacInnes \inst{3} }

\institute[VFU] % (optional)
{
  \inst{1}%
  Department of Demography\\
  University of California, Berkeley
  \and
  \inst{2}%
  Vienna Institute of Demography
  \and
  \inst{3}
  School of Social and Political Science\\
  University of Edinburgh
}
 
\date[Dec 2014] % (optional)
{New Measures of Age and Ageing, Dec 2014}

\begin{document}

\begin{frame}
\titlepage % Print the title page as the first slide
\end{frame}

\section{Background} % A subsection can be created just before a set of slides
% with a common theme to further break down your presentation into chunks

\begin{frame}
\frametitle{Formal to empirical}
This work was inspired by some formal results that begged the question
\begin{block}{The question}
What kinds of demographic phenomena vary in informative and empirically regular
ways by remaining years of life?
\end{block}
\end{frame}

%------------------------------------------------

\begin{frame}
\frametitle{Formal to empirical}

\end{frame}

%------------------------------------------------

\begin{frame}
\frametitle{Blocks of Highlighted Text}
\begin{block}{Block 1}
Lorem ipsum dolor sit amet, consectetur adipiscing elit. Integer lectus nisl, ultricies in feugiat rutrum, porttitor sit amet augue. Aliquam ut tortor mauris. Sed volutpat ante purus, quis accumsan dolor.
\end{block}

\begin{block}{Block 2}
Pellentesque sed tellus purus. Class aptent taciti sociosqu ad litora torquent per conubia nostra, per inceptos himenaeos. Vestibulum quis magna at risus dictum tempor eu vitae velit.
\end{block}

\begin{block}{Block 3}
Suspendisse tincidunt sagittis gravida. Curabitur condimentum, enim sed venenatis rutrum, ipsum neque consectetur orci, sed blandit justo nisi ac lacus.
\end{block}
\end{frame}

%------------------------------------------------

\begin{frame}
\frametitle{Multiple Columns}
\begin{columns}[c] % The "c" option specifies centered vertical alignment while the "t" option is used for top vertical alignment

\column{.45\textwidth} % Left column and width
\textbf{Heading}
\begin{enumerate}
\item Statement
\item Explanation
\item Example
\end{enumerate}

\column{.5\textwidth} % Right column and width
Lorem ipsum dolor sit amet, consectetur adipiscing elit. Integer lectus nisl, ultricies in feugiat rutrum, porttitor sit amet augue. Aliquam ut tortor mauris. Sed volutpat ante purus, quis accumsan dolor.

\end{columns}
\end{frame}

%------------------------------------------------
\section{Second Section}
%------------------------------------------------


%------------------------------------------------


%------------------------------------------------

\begin{frame}
\frametitle{Figure}

\begin{figure}
\includegraphics[width=0.8\linewidth]{Figures/CaseCountFemales}
\end{figure}
\end{frame}

%------------------------------------------------


%------------------------------------------------


%------------------------------------------------


%----------------------------------------------------------------------------------------

\end{document}





